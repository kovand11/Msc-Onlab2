\documentclass[]{article}

\usepackage[magyar]{babel}
\usepackage[utf8]{inputenc}

%opening
\title{Pénzügyi big data szerkesztése a piacvezető Q nyelv segítségével}
\author{András Kovács}

\begin{document}

\maketitle

\begin{abstract}
	


\end{abstract}

\section{Modell}


\paragraph{Piaci szereplők}

\begin{description}
	\item[Type A] Description A
	\item[Type B] Description B
\end{description}

Poisson eloszlás és magyarázat. 

\section{Implementáció}

\begin{description}
	\item[Ábra az adatsruktúráról] 
	\item[Ábra a feldolgozásról]
\end{description}

\subsection{KDB+}

A projekt alapját képező adathalmaz, mintegy 2.5 terrabájt, egy KDB+ szerveren volt tárolva. Ennek tartalma több évnyi, amerikai tőzsdéken jegyzett vállalatok kereskedési adatai. (T\&Q)
A fájlok egy része ASCII másik rész bináris formátumban volt elérhető. Az évek során a fájlformátum kis mértékben változott.
\subsection{Python}
Az adatbázis lekérdezések automatizálására, és a feladat azon részeire amiket adatbázis-műveletekkel nem lehetett megoldani, egy Python feldolgozó szkriptet hoztam létre. 
\subsection{Wolfram Mathematica}
A modell paramétereinek Maximum Likelihood becslőének kiszámításához a Wolfram Research Mathematica szoftverét használtam. Előnye legfőképpen az volt, hogy szimbolikus algebrával dolgozik, numerikus instabilitás nem jelent problémát.


\section{Eredmények}

\section{További lehetőségek}


\end{document}


